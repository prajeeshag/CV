% !TEX root = ./prajeesh_cv.tex
% -- Encoding UTF-8 without BOM
% -- XeLaTeX => PDF (BIBER)
%
\documentclass[]{cv-style}          % Add 'print' as an option into the square bracket to remove colours from this template for printing. 
                                    % Add 'espanol' as an option into the square bracket to change the date format of the Last Updated Text

\sethyphenation[variant=british]{english}{} % Add words between the {} to avoid them to be cut 
\usepackage{outlines}
\begin{document}

\header{Prajeesh }{A G}           % Your name
\lastupdated

%----------------------------------------------------------------------------------------
%	SIDEBAR SECTION  -- In the aside, each new line forces a line break
%----------------------------------------------------------------------------------------

\begin{aside}
%
\section{contact}
CCCR, IITM
Pashan, Pune,
India, 411008
~
+91 705 7253778
~
prajeeshag@gmail.com
%
\section{languages}
English
Hindi
Malayalam
%
\section{Programming}
    Python, FORTRAN
    C, Bash, HPC, MPI, OpenMP, NCL, Matlab, ferret, CDO, NCO, Cylc, Xarray, Matplotlib, Jupyter-Notebook, Django, JavaScript, HTML, CSS, \LaTeX{}
%
\end{aside}

\section{career Summary}
  \vspace{-0.2cm}
Research scientist specialised in climate model development. Over 7+ years of experience in developing and upgrading various numerical models and software used for scientific research. Skilled in numerical methods, high performance computing, parallel programming, FORTRAN, python and data visualization. Has strong analytical and problem solving skills. Always keen to explore new areas to expand my knowledge and help solve important problems for society.

\section{experience}
\begin{exprlist}
  \expr
  {2013--2016}
  {Scientist B}
  {Centre for Climate Change Research, IITM Pune, India}
  \expr
  {2016--2020}
  {Scientist C}
  {Centre for Climate Change Research, IITM Pune, India}
  \expr
  {2020--Now}
  {Scientist D}
  {Centre for Climate Change Research, IITM Pune, India}
\end{exprlist}
\begin{outline}
    \1 \textsc{Earth System Model (\,IITM-ESM)\, Development}
        \2 Development of a new spectral dynamical core for IITM-ESM which enables a 2D domain decomposition in both spectral and the grid point domain, significantly improving the scalability and throughput. Parallel FFT is implemented by the FFTW library followed by a parallel Legendre transform.
        \2 Solving the top of the atmosphere radiation imbalance in IITM-ESM. Energy budget calculations were performed for every process in the model to narrow down the source of errors and appropriate changes were made to fix the errors.
        \2 Developed a framework for creating perturbed initial condtions for conducting targeted large ensemble climate sensitivity experiments and forecast runs.
        \2 Implemented a fractional grid for computing surface fluxes and resolved the inconsistency in the surface flux computation and flux-transfer over the sea-ice and land-ocean boundaries, which resulted in significant improvements in the sea-ice simulations.
        \2 Development of a parallel I/O manager using Netcdf-Fortran which provide a simple and efficient way to get diagnostic outputs and to override data of a model field from a Netcdf file.
        \2 Implemented concurrent coupling using FMS coupler to increase the throughput of the model.
        \2 Implementing a low resolution version of IITM-ESM.
        \2 Porting of model code to different HPC environments.
    \1 \textsc{IITM-ESM CMIP6 simulations}
        \2 Developing post-processing software necessary for CMIP6 data requirements.
        \2 Implementing CMIP6 input forcing such as aerosols and land-use land-cover data for IITM-ESM.
        \2 Implementing workflow for various CMIP6 experiments and data processing.
    \1 \textsc{Short range weather prediction}
        \2 Developing a high resolution (~5 km) short range forecast system based of GFSv14 with a new Triangular-Cubic-Octahedral (TCO) grid and revised horizontal diffusion scheme.
        \2 Coupling GFSv14 with Ocean model for investigating the predictability of cyclones and depression over short and extended range forecasts.
% \pagebreak
    \1 \textsc{ODTM development}
        \2 ODTM is an isopycnal reduced gravity dynamics-thermodynamic model for the tropical oceans. The MPI parallelisation of the model was developed using the FMS (Flexible Modelling System).
        \2 Coupling of Nemuro eco-system model with ODTM to study the ecosystem response in tropical Indian Ocean.
    
    \1 \textsc{Indian Ocean Dipole (IOD) variations in a warming climate and associated linkages to monsoon}
        \2 Understanding of the Indian Ocean Dipole connections with the South Asian Monsoon in observations and IITM ESM.
        \2 To understand future changes in IOD and its linkages to monsoon
    
    \1 \textsc{Other Activities}
        \2 To provide consulting services for setting up complex experiments using various numerical models, optimal use of scientific software, parallel algorithm development and workflow optimization around HPC platform
        \2 Investigating code written by peers to identify, debug and troubleshoot issues related to scientific algorithms.
        \2 Teaching sessions for programming languages and other scientific software to the research personnel.
\end{outline}

\begin{exprlist}
  \expr
  {2010--2011}
  {Junior Research Fellow}
  {Dept. Of Ocean Engineering, IIT Kharagpur, India}
\end{exprlist}
\begin{itemize}
    \item Setting up a regional wave model (\,SWAN)\, with unstructured grid for the gulf of Kutch.
\end{itemize}

\begin{exprlist}
  \expr
  {2020--Now}
  {Web developer}
  {Freelance}
\end{exprlist}
\begin{itemize}
    \item Developed a football League website using Django with the facility of team and player registration, player transfer, match scheduler, on-ground score and match details entering, player and team statistics, and standings table.
    \item Developed an e-commerce website using Django-Oscar and  a news blog using Wagtail CMS.
\end{itemize}

\section{skills}
    \begin{itemize}
        \item Possess expertise in understanding the requirements of the scientists and research personnel and writing small/big complex modules in FORTRAN, C, Python, Bash and other scripting languages.
        \item Solid experience of designing and developing software using concepts of high performance computing, MPI and OpenMP parallelisation and advanced I/O using parallel Netcdf.
        \item Sound understanding in the coupling technologies used in climate models and experience in coupling component models to build coupled climate models.
        \item Expertise in tools used for climate data manipulation and visualization such as CDO, NCO, Xarray, Matplotlib, NCL, Jupyter notebook etc.
        \item Experience in working with workflow management tools such as Cylc.
        \item Experience in working with debugging and profiling tools such as DDT, Totalview and Valgrind.
        \item Experience in developing and upgrading atmospheric, oceanic, and climate system models using software frameworks such as Flexible Modelling System (FMS).
        \item Good understanding and experience in Unix/Linux operating systems.
        Experience in web development using Django. Proficiency in JavaScript, HTML, CSS and deploying web applications on cloud servers.
        \item Intermediate experience in constraint programming (optimization) using OR-Tools and AI/ML applications.
        \item Adept in troubleshooting software problems, recommending alternative solutions for the same.
        \item Teaching Experience in training courses on FORTRAN, Python, and numerical models.
        \item Fast learner with good mathematical and logical reasoning skills.
        \item Team player, and possess outstanding problem solving and interpersonal skills.
    \end{itemize}


\section{education}
\begin{entrylist}
%------------------------------------------------
\entry
{2008--2010}
{M.Sc. {\normalfont in Oceanography}}
{Cochin University of Science And Technology, Kochi, India}
{Main subjects: Fluid Dynamics, Ocean Circulation, Numerical Modelling of Ocean. \\
Dissertation: ”Effect of tidal currents on wind waves in the Gulf of Kutch region, Gujarat”}
%------------------------------------------------
\entry
{2004--2007}
{B.Sc. {\normalfont in Physics}}
{University of Calicut, Calicut, India}
{Main subjects: Physics, Mathematics, Computer Science}
%------------------------------------------------
\end{entrylist}

%----------------------------------------------------------------------------------------
%	OTHER QUALIFICATIONS SECTION
%----------------------------------------------------------------------------------------

\section{other qualifications}

\begin{entrylist}
%------------------------------------------------
\entry
{2011-2012}
{Coursework in Earth System Science and Climate}
{Centre for Advanced Training, IITM Pune, India}
{Dissertation: Study on declining of Monsoon Depressions in Changing Climate}
\end{entrylist}

%----------------------------------------------------------------------------------------
%	AWARDS SECTION
%----------------------------------------------------------------------------------------

\section{other info}
\begin{itemize}
    \item \emph{University first rank holder, MSc Oceanagraphy 2010, CUSAT}
    \item \emph{Cleared CSIR-UGC-NET JRF Scholarship in 2010}
\end{itemize}

\section{publications}
\begin{itemize}
    \item \textbf{Prajeesh AG}, Swapna P, Krishnan R, Ayantika DC, Sandeep N, Manmeet S, Aditi M, Sandip I. The Indian summer monsoon and Indian Ocean Dipole connection in the IITM Earth System Model (IITM-ESM). Climate Dynamics. 2021 Oct 15:1-21.
    \item \textbf{Prajeesh AG}, Ashok K, Rao DB. Falling monsoon depression frequency: A Gray-Sikka conditions perspective. Scientific reports. 2013 Oct 21;3(1):1-8.
    \item Swapna P, Krishnan R, Sandeep N, \textbf{Prajeesh AG}, Ayantika DC, Manmeet S, Vellore R. Long‐Term Climate Simulations Using the IITM Earth System Model (IITM‐ESMv2) With Focus on the South Asian Monsoon. Journal of Advances in Modeling Earth Systems. 2018 May;10(5):1127-49.
    \item Swapna P, Roxy MK, Aparna K, Kulkarni K, \textbf{Prajeesh AG}, Ashok K, Krishnan R, Moorthi S, Kumar A, Goswami BN. The IITM earth system model: transformation of a seasonal prediction model to a long-term climate model. Bulletin of the American Meteorological Society. 2015 Aug;96(8):1351-67.
    \item Singh S, Valsala V, \textbf{Prajeesh AG}, Balasubramanian S. On the variability of Arabian Sea mixing and its energetics. Journal of Geophysical Research: Oceans. 2019 Nov;124(11):7817-36.
    \item Sandeep N, Swapna P, Krishnan R, Farneti R, \textbf{Prajeesh AG}, Ayantika DC, Manmeet S. South Asian monsoon response to weakening of Atlantic meridional overturning circulation in a warming climate. Climate Dynamics. 2020 Apr;54(7):3507-24
    \item Ayantika DC, Krishnan R, Singh M, Swapna P, Sandeep N, \textbf{Prajeesh AG}, Vellore R. Understanding the combined effects of global warming and anthropogenic aerosol forcing on the South Asian monsoon. Climate Dynamics. 2021 Mar;56(5):1643-62.
    \item Krishnan R, Swapna P, Vellore R, Narayanasetti S, \textbf{Prajeesh AG}, Choudhury AD, Singh M, Sabin TP, Sanjay J. The IITM earth system model (ESM): development and future roadmap. InCurrent Trends in the Representation of Physical Processes in Weather and Climate Models 2019 (pp. 183-195). Springer, Singapore.
    \item Singh M, Krishnan R, Goswami B, Choudhury AD, Swapna P, Vellore R, \textbf{Prajeesh AG}, Sandeep N, Venkataraman C, Donner RV, Marwan N. Fingerprint of volcanic forcing on the ENSO–Indian monsoon coupling. Science advances. 2020 Sep 1;6(38):eaba8164.
    \item Terray, P., K.P. Sooraj, S. Masson, R.P.M. Krishna, G. Samson and \textbf{AG Prajeesh} (2018) "Towards a realistic simulation of boreal summer tropical rainfall climatology in state-of-the art coupled models : role of the background snow-free albedo", Climate Dynamics, Vol. 50: 3413–3439
\end{itemize}

\end{document}